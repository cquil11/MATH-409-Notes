\documentclass{article}
\usepackage[utf8]{inputenc}
\usepackage{amssymb}
\usepackage{amsthm}
\usepackage{enumerate}
\usepackage{mathtools}
\usepackage{nicefrac}
\usepackage{tikz}
\usepackage[inline,shortlabels]{enumitem}
\usepackage{hyperref}
\usepackage{imakeidx}
\usepackage{thmtools}
\usepackage[framemethod=TikZ]{mdframed} % Needed for the box and shading
\usepackage{xcolor} % To define custom colors

\definecolor{lightgray}{gray}{0.9}


\title{Analysis on the Real Line}
\author{Cam Quilici - Instructor Dr. Andrea Bonito}
\date{Fall 2023}

\setlist[itemize,1]{topsep=0pt,itemsep=2em,partopsep=0.5em,parsep=0pt}

\newcommand{\Z}{\mathbb{Z}}
\newcommand{\C}{\mathbb{C}}
\newcommand{\Q}{\mathbb{Q}}
\newcommand{\R}{\mathbb{R}}
\newcommand{\N}{\mathbb{N}}
\newcommand{\seq}[2]{(#1_{#2})_{#2 \in \N}}
\newcommand{\sseq}[3]{(#1_{#2_#3})_{#3 \in \N}}
\newcommand{\set}[2]{\{ #1 \mid #2 \}}
\newcommand{\sg}[1]{\langle#1\rangle}
\newcommand{\mylim}[2]{\lim_{#1 \to #2}}
\newcommand{\mylimmm}[2]{\lim\limits_{#1 \to #2}}
\newcommand{\bigslant}[2]{{\raisebox{.2em}{$#1$}\left/\raisebox{-.2em}{$#2$}\right.}}
\newcommand{\?}{\stackrel{?}{=}}
\def\checkmark{\tikz\fill[scale=0.4](0,.35) -- (.25,0) -- (1,.7) -- (.25,.15) -- cycle;}
\newcommand{\smallblacksquare}{\rule{0.5em}{0.5em}}
\newcommand\encircle[1]{%
  \tikz[baseline=(X.base)] 
    \node (X) [draw, shape=circle, inner sep=0] {\strut #1};}

\DeclareMathOperator{\ord}{ord}
\DeclareMathOperator{\norm}{Nm}

\makeatletter
\renewcommand\thmt@listnumwidth{4.3em}
\makeatother

\newmdtheoremenv[
    nobreak=true,
    backgroundcolor=lightgray,
    skipabove=1em,
    innertopmargin=.5em, % space added at the top
    innerbottommargin=.5em, % space added at the bottom
    innerleftmargin=1em, % space added on the left
    innerrightmargin=1em, % space added on the right
]{theorem}{Theorem}[subsection]
\newmdtheoremenv[
    nobreak=true,
    skipabove=1em,
    innertopmargin=.5em, % space added at the top
    innerbottommargin=.5em, % space added at the bottom
    innerleftmargin=1em, % space added on the left
    innerrightmargin=1em, % space added on the right
]{lemma}[theorem]{Lemma} % shares counter with theorem
\newmdtheoremenv[
    nobreak=true,
    skipabove=1em,
    innertopmargin=.5em, % space added at the top
    innerbottommargin=.5em, % space added at the bottom
    innerleftmargin=1em, % space added on the left
    innerrightmargin=1em, % space added on the right
]{proposition}{Proposition}[subsection] % Proposition environment

\theoremstyle{definition} % This style is typically used for definitions, examples, etc.
\newtheorem{definition}[subsection]{Definition} % Numbered by subsection

\makeindex

\begin{document}

\maketitle

\vspace{-0.3in}
\noindent
\rule{\linewidth}{0.4pt}

\tableofcontents

\newpage

\section{The Real Numbers}

\subsection{Introduction}

\begin{itemize}
    \item[]
    \begin{lemma}
        The equation $x^2 - 2 = 0$ has no solution in $\Q$.
        \index{Lemma \thelemma}
    \end{lemma}
    \begin{proof}
        By contradiction, assume there exists some $\nicefrac{p}{q} \in \Q$, $p, q \in \N$, $q \neq 0$ such that
        $$\left(\frac{p}{q}\right)^2 - 2 = 0. \quad (*)$$
        Without loss of generality, we can assume that the greatest common divisor between $p$ and $q$ is 1. We rewrite (*) as $p^2 = 2q^2$ which implies that $p^2$ is even. This means that $p$ is even as well.
    \end{proof}
    \item We say that $\N$ is well-ordered, but not $\Q$ since $\Q$ does not have a least element.
    \item[]
    \begin{proposition}
        There is no natural number such that $0 < n < 1$.
    \end{proposition}
    \begin{proof}
        Left as an exercise.
    \end{proof}
\end{itemize}

\subsubsection{Axioms of the Real Numbers}

\begin{itemize}
    \item Binary operations:
    \begin{itemize}[label=$\smallblacksquare$]
        \item ($+$): $\R \times \R \mapsto \R \qquad x, y \in \R, \ x + y \in \R$,
        \item ($\cdot$): $\R \times \R \mapsto \R \qquad x, y \in \R, \ x \cdot y \in \R$.
    \end{itemize}
    \item We have axioms for the real numbers as follows:
    \begin{enumerate}[label=\Roman*]
        \item Algebraic Axioms
        \begin{enumerate}[label=(\roman*)]
            \item Associativity: For $x, y, z \in \R$, we have
            \begin{align*}
                x + (y + z) &= (x + y) + z \\
                x \cdot (y \cdot z) &= (x \cdot y) \cdot z
            \end{align*}
            \item Commutativity: For $x, y \in \R$, we have
            \begin{align*}
                x + y &= y + x \\
                x \cdot y &= y \cdot x
            \end{align*}
            \item Distributivity: For $x, y, z \in \R$, we have
            \begin{align*}
                x \cdot (y + z) &= x \cdot y + x \cdot z
            \end{align*}
            \item Identity:
            \begin{itemize}[label=$\smallblacksquare$]
                \item Addition: There exists some $0 \in \R$ such that for all $x \in \R$, $x + 0 = x$.
                \item Multiplication: There exists some $1 \in \R$ such that for all $x \in \R$, $x \cdot 1 = x$.
            \end{itemize} 
            \item Inverses:
            \begin{itemize}[label=$\smallblacksquare$]
                \item Addition: For all $x \in \R$, there exists some $y \in \R$ such that
                $$x + y = 0 \iff y = -x.$$
                \item Multiplication: For all $x \in \R$, there exists some $y \in \R$ such that
                $$x \cdot y = 1 \iff y = \frac{1}{x} \iff y = x^{-1}.$$
            \end{itemize}
        \end{enumerate}
        \item Ordering
        \begin{enumerate}[label=(\roman*)]
            \item For some $x, y, z \in \R$, we have
            $$x \leq y \implies x + z \leq y + z.$$
            \item For some $x, y \in \R$, we have
            $$0 \leq x, 0 \leq y \implies 0 \leq x \cdot y.$$
            \item For some $x, y, z \in \R$, we have
            $$x \leq y, y \leq z \implies x \leq z.$$
            \item For some $x, y \in \R$, we have
            $$x \leq y, y \leq x \implies x = y.$$
            \item For some $x, y \in \R$, we have
            $$x \neq y \implies x \leq y \text{ or } y \leq x.$$
        \end{enumerate}
        \item No Hole (not satisfied by $\Q$) \\
        For any non-empty subset $X$ of $\{x > 0\}$, there exists some $a \in \R$ such that
        \begin{enumerate}[label=(\arabic*)]
            \item $a \leq x \ \forall x \in X$, and
            \item For all $\varepsilon > 0 (\varepsilon \neq 0, 0 \leq \varepsilon)$, there exists some $x_\varepsilon$ such that $x_\varepsilon - a \leq \varepsilon$.
        \end{enumerate}
    \end{enumerate}
\end{itemize}

\subsubsection{Bounded Intervals}

\begin{itemize}
    \item Given $a < b$, the set $X$ of real numbers is a bounded interval if it is in the form:
    \begin{itemize}[label=\smallblacksquare]
        \item Open interval with endpoints $a, b$
        $$\set{x \in X}{a < x < b} = (a, b).$$
        \item Closed intervals
        $$\set{x \in X}{a \leq x \leq b} = [a, b].$$
        \item Half-open intervals
        $$\set{x \in X}{a < x \leq b} = (a, b].$$
    \end{itemize}
\end{itemize}

\subsection{Supremum and Infimum}

\begin{itemize}
    \item[] 
    \begin{definition}
        Let $X \subseteq \R$ be non-empty. A number $m \in \R$ (not necessarily in $X$) is a \textbf{lower bound} for $X$ if for all $x \in X$, $m \leq x$.
    \end{definition}
    If a set has a lower bound, it is called \textbf{bounded below} (for instance, $\Z$).
    \item \textbf{Remark}. $X$ either has no lower bounds or infinitely many.
    \item[]
    \begin{definition}
        Let $X \subseteq \R$ be non-empty. A number $s \in \R$ (not necessarily in $X$) is called an \textbf{infimum} of the set $X$ if it is
    \begin{enumerate}[label=(\arabic*)]
        \item A lower bound and
        \item For all other lower bounds $r_i$ of $X$, $s \geq r_i$.
    \end{enumerate}
    The infimum is hence known as the ``greatest lower bound." We denote ``$r$ is the infimum of $X$" as
    $$r = \inf(X).$$
    \end{definition}
    \item[]
    \begin{definition}
        Let $X \subseteq \R$ be non-empty. A number $s \in \R$ (not necessarily in $X$) is called a \textbf{supremum} of the set $X$ if it is
    \begin{enumerate}[label=(\arabic*)]
        \item An upper bound and
        \item For all other upper bounds $r_i$ of $X$, $s \leq r_i$.
    \end{enumerate}
    The supremum is hence known as the ``least upper bound." We denote ``$r$ is the supremum of $X$" as
    $$r = \sup(X).$$
    \end{definition}
    \item[]
    \begin{theorem}
         Let $X \subseteq \R$ be non-empty. $X$ has an upper bound if and only if $X$ has a finite supremum and the finite supremum is unique.
    \end{theorem}
    \begin{proof}
        If $X$ has a supremum then that supremum is an upper bound for $X$. Assume that $b$ is an upper bound of $X$. Let $\overline{X} = \set{b + 1 - y}{y \in X}$, then $\overline{X}$ is non empty and $\overline{X} \subset \{x > 0\}$ (there is a 1 so it is surely greater than 0). \\\\
        So, by Axiom 3, there is a real number $a$ such that
        \begin{enumerate}[label=(\arabic*)]
            \item If $z \in \overline{X}$, then $a \leq z$ (i.e., $a$ is a lower bound for $\overline{X}$) and
            \item For every $\varepsilon > 0$, there exists a $z_\varepsilon \in \overline{X}$ such that
            $$z_\varepsilon - a \leq \varepsilon.$$
        \end{enumerate}
        \textbf{Claim}. $b_1 = b + 1 - a$ is the supremum for $X$. \\\\
        There are two things to check:
        \begin{enumerate}[label=(\arabic*)]
            \item Is $b_1$ an upper bound for $X$? Well, if $y \in X$, then $a \leq b + 1 - y$ and we have $y \leq b + 1 - a = b_1$.
            \item Is $b_1$ the least upper bound? \\\\
            Let $M$ be some upper bound for $X$. Let us show that $b_1 \leq M$. By way of contradiction, assume that $M < b_1 = b + 1 - a$. Let $\varepsilon = b_1 - M > 0$. \\\\
            By Axiom 3, there exists some $z \in \overline{X}$ such that $z - a \leq \varepsilon$. We have
            $$z = b + 1 - y, y \in X, \qquad \varepsilon = b_1 - M = b + 1 - a - M.$$
            So, supposedly $z - a = b + 1 - y - a \leq b + 1 - a - M$ with $M \leq y$ (this is a contradiction since $M$ is assumed to be an upper bound, it cannot be smaller than $y$) and therefore $b_1 \leq M$ and $b_1 = \sup(X)$.
        \end{enumerate}
    \end{proof}
    \item \textbf{Archimedean Axiom} (satisfied in $\R$). For every $x > 0, y \geq 0$, there exists $n \in \N_*$ such that
    $$n \cdot x > y.$$
    \begin{proof}
        By way of contradiction, assume that $n \cdot x \leq y \ \forall n \in \N_*$ (for some $x > 0, y \geq 0)$. Then let $S = \set{n \cdot x}{n \in \N_*} \subset \R$. Clearly $S$ is bounded above by $y$ and $S$ is non-empty. This implies that $\sup(S)$ exists and $n \cdot x \leq \sup(s) \ \forall n \in \N_*$. This all implies that $(m + 1) \cdot x \leq \sup(S) \ \forall m \in \N_*$
    \end{proof}
    \item[]
    \begin{lemma}
        Let $X \subset \R$ be non-empty and bounded above. Then for every $\varepsilon > 0$, there exists some $x_\varepsilon$ such that
    $$\sup(X) - \varepsilon < x_\varepsilon \leq \sup(X).$$
    \end{lemma}
    \begin{proof}
        Since $\sup(X)$ is an upper bound for $X$, then we know $x \leq \sup(X) \ \forall x \in X$. This implies the second part of the inequality. \\\\
        To prove the first part of the inequality, assume by way of contradiction that there exists some $\varepsilon_0 > 0$ such that for every $x \in X$, $\sup(x) - \varepsilon_0 \geq x$. However, this is not possible since $\sup(x)$ is an upper bound, i.e., subtract anything from it and it will be less than $x$.
    \end{proof}
    \item[]
    \begin{lemma}
        Let $x, y \in \R$ with $y > x$, then there exists some $r \in \Q$ such that $x < r < y$.
    \end{lemma}
    \begin{proof}
        Take $y - x > 0$, the Archimidean principle guarantees the existence of $n \in \N$ such that
        $$n \cdot (y - x) > 1 \iff y - x > \frac{1}{n}.$$
        Recall,
        \begin{align*}
            \lfloor x \rfloor &\leq x \leq \rfloor x \lfloor + 1 \\
            \lfloor nx \rfloor &\leq \frac{nx}{n} < \frac{nx + 1}{n}.
        \end{align*}
        Let $r := \nicefrac{(\lfloor nx \rfloor + 1)}{n} \in \Q$. Then we have
        \begin{align*}
            x = \frac{nx}{n} &< \frac{\lfloor nx \rfloor + 1}{n} \\
            &\leq \frac{nx + 1}{n} \\
            &= x + \frac{1}{n} \\
            &= x + y - x \\
            &< y.
        \end{align*}
    \end{proof}
    \item \textbf{Corollary.} Let $p, q \in \Q$ with $p < q$. There exists some $z \in \R \ \symbol{92} \ \Q$ such that $p < z < q$.
    \begin{proof}
        By Lemma,
        \begin{align*}
            \frac{p}{c} < \frac{q}{c} &\iff \frac{p}{c} < \underbrace{r}_{\in \Q} < \frac{q}{c} \\
            &\iff p < \underbrace{r \cdot c}_{\in \R \ \symbol{92} \ \Q} < q.
        \end{align*}
    \end{proof}
\end{itemize}

\subsubsection{Absolute Value}

\begin{itemize}
    \item[]
    \begin{definition}
        For some $x \in \R$, the \textbf{absolute value} function is defined as follows:
    \[ |x| := \begin{cases}
        -x & \text{, if } x < 0 \\
        x & \text{, if } x \geq 0.
    \end{cases} \]
    \end{definition}
    \item[]
    \begin{proposition}
        Let $x, a \in \R$. Then we have
    \begin{enumerate}[label=(\roman*)]
        \item $|x| = 0 \iff x = 0$.
        \item Let $a > 0$, then
        \begin{itemize}[label=\smallblacksquare]
            \item $|x| < a \iff -a < x < a$
        \end{itemize}
    \end{enumerate}
    \end{proposition}
\end{itemize}

\section{Sequences}

\subsection{Convergence}

\begin{itemize}
    \item[]
    \begin{definition}
        A sequence of real numbers $\seq{x}{n}$ is said to be \textbf{convergent} if there exists $x \in \R$ such that $\forall \varepsilon > 0 \ \exists N(\varepsilon) \in \N$ such that
    $$|x - x_n| \leq \varepsilon \ \forall n \geq N(\varepsilon).$$
    Furthermore, we write $\lim_{n \to \infty} x_n = x$.
    \end{definition}
    \item \textbf{Example}. Show that $\lim_{n \to \infty} 1 + \nicefrac{1}{(n + 1)} = 1$. We want
    $$|x - x_n| = \left|1 - \left(1 + \frac{1}{n + 1}\right)\right| = \left|\frac{1}{n + 1}\right| = \frac{1}{n + 1} \leq \varepsilon \ \forall \varepsilon > 0.$$
    Let $\varepsilon > 0$, set $N(\varepsilon) = \lfloor \nicefrac{1}{\varepsilon} \rfloor = N(\varepsilon)$. This holds for
    \begin{align*}
        |x - x_n| &= \frac{1}{n + 1} \leq \frac{1}{\lfloor \frac{1}{\varepsilon} \rfloor + 1} \\
        &\leq \frac{1}{\frac{1}{\varepsilon}} \leq \varepsilon.
    \end{align*}
    \item[]
    \begin{lemma}
        A sequence of real numbers has \textbf{at most} one limit.
    \end{lemma}
    \begin{proof}
        Assume $l_1, l_2 \in \R$ are two limits. We will show that $l_1 = l_2$ by showing that for all $\varepsilon > 0$, $|l_1 - l_2| \leq \varepsilon$. Let $\varepsilon > 0$.
        \begin{enumerate}[label=(\arabic*)]
            \item $\seq{x}{n}$ converges to $l_1$, i.e., for all $\varepsilon_1 > 0 \ \exists N_1(\varepsilon_1)$ such that
            $$|x_n - l_1| \leq \varepsilon_1 \ \forall n \geq N_1(\varepsilon_1),$$
            \item $\seq{x}{n}$ converges to $l_2$, i.e., for all $\varepsilon_2 > 0 \ \exists N_2(\varepsilon_2)$ such that
            $$|x_n - l_2| \leq \varepsilon_2 \ \forall n \geq N_2(\varepsilon_2).$$
        \end{enumerate}
        Let $\varepsilon > 0$ and set $\varepsilon_1 - \nicefrac{\varepsilon}{2}, \varepsilon_2 = \nicefrac{\varepsilon}{2}$. Then we have
        \begin{align*}
            |l_1 - l_2| &= |l_1 - x_n + x_n - l_2| \\
            &\leq |l_1 - x_n| + |l_2 - x_n| \leq \varepsilon
        \end{align*}
        where $|l_1 - x_n| \leq \varepsilon_1 \ \forall n \geq N_1$ and $|l_2 - x_n| \leq \varepsilon_2 \  n \geq N_2 \ n \geq \max\{N_1, N_2\}$. Altogether, this implies that $l_1 = l_2$.
    \end{proof}
    \item \textbf{Example of Divergent Sequence}. Take $x_n = n \cdot \sin(n \cdot \nicefrac{\pi}{2})$. By way of contradiction, assume that $\seq{x}{n}$ is converging to some $x \in \R$. Say
    \begin{align*}
        x_{p_n} &= x_{4n + 1} = (4n + 1) \cdot \underbrace{\sin((4n + 1) \cdot \frac{\pi}{2})}_{= 1} \\
        &= 4n + 1 \\
        x_{q_n} &= x_{4n + 5} = (4n + 5) \cdot \underbrace{\sin((4n + 5) \cdot \frac{\pi}{2})}_{= 1} \\
        &= 4n + 5
    \end{align*}
    which implies that $|x_{p_n} - x_{x_n}| = 4$. From the contradiction assumption, for $\varepsilon = \nicefrac{1}{2}$ there exists $N(\nicefrac{1}{2})$ such that
    $$|x_n - x| \leq \frac{1}{2} \ \forall n \geq N(\frac{1}{2}).$$
    Take $n$ such that $4n + 1 \geq N(\nicefrac{1}{2})$. Then we have
    \begin{align*}
        |x_{p_n} - x| &\leq \frac{1}{2} \\
        |x_{q_n} - x| &\leq \frac{1}{2} \\
        4 &= |x_{p_n} - x_{q_n}| = |x_{p_n} - x + x - x_{q_n}| \\
        &\leq |x_{p_n} - x| + |x_{q_n} - x| \\
        &\leq \frac{1}{2} + \frac{1}{2} \leq 1
    \end{align*}
    but this is a contradiction, so $\seq{x}{n}$ must be divergent.
    \item[]
    \begin{lemma}
        Every convergent sequence is bounded.
    \end{lemma}
    \begin{proof}
        Let $\seq{x}{n}$ be a convergent sequence, that is, there exists some $x \in \R$ such that for all $\varepsilon > 0$, there exists some $N(\varepsilon) \in \N$ and
        $$|x - x_n| \leq \varepsilon \ \forall n \geq N(\varepsilon).$$
        In particular, take for $\varepsilon = 1$, there is $N(1)$ such that
        $$|x_n - x| \leq 1 \ \forall n \geq N(1).$$
        Recall the reverse triangle inequality,
        \begin{alignat*}{2}
        |x_n| - |x| &\leq |x - x_n| &&\leq 1 \quad n \geq N(1) \\
        \implies \, |x_n| &\leq 1 + |x| \quad &&n \geq N(1)
        \end{alignat*}
        Let $M = \max\{|x_0|, |x_1|, \ldots, |x_{N(1) - 1}|, 1 + |x|\}$. Then certainly
        \begin{align*}
            |x_n| &\leq M \qquad n \leq N(1) \\
            |x_n| &\leq M \qquad n \geq N(1) \\
            \implies |x_n| &\leq M \qquad \forall n \in \N.
        \end{align*}
    \end{proof}
    \item \textbf{Recall}. A sequence $(x_n)_{n \in \N}$ convergent $\implies \exists M \in \N$ such that $|x_n| \leq M \ \forall n \in \N$.
    \item[]
    \begin{definition}
        A sequence $\seq{x}{n}$ is said to be
    \begin{itemize}[label=\smallblacksquare]
        \item \textbf{Increasing} if $x_m \geq x_n$ when $m \geq n$,
        \item \textbf{Decreasing} if $x_m \leq x_n$ when $m \geq n$,
        \item \textbf{Monotone} if $\seq{x}{n}$ is increasing or decreasing.
    \end{itemize}
    \end{definition}
    \item[]
    \begin{lemma}
        Any increasing sequence that is bounded above is convergent.
    \end{lemma}
    \begin{proof}
        Let $X = \set{x_n}{n \in \N}$. Then $x_n \leq M, n \in \N \implies X$ bounded above $\implies x = \sup(X)$. \\\\
        We want to show that $\seq{x}{n}$ converges to $x = \sup(X)$. This means (by definition) that for every $\varepsilon > 0$, there exists $N(\varepsilon) \in \N$ such that
        $$|x_n - x| \leq \varepsilon \ \forall n \geq N(\varepsilon).$$
        Note that $x = \sup(X)$ implies two important things:
        \begin{enumerate}[label=(\roman*)]
            \item $x_n \leq x \ \forall n \in \N$,
            \item $\forall \varepsilon > 0, \ \exists x_\varepsilon \in X$ such that $x - x_\varepsilon \leq \varepsilon$ by the characterization of supremum. Note, we can equivalently replace $x_\varepsilon$ with $x_{n_\varepsilon} \in \N$ to obtain $\forall \varepsilon > 0, \ \exists x_{n_\varepsilon} \in X$ such that $x - x_{n_\varepsilon} \leq \varepsilon$.
        \end{enumerate}
        So, let $\varepsilon > 0$, compute
        \begin{align*}
            |x_n - x| &= x - x_n \\
            &\leq x - x_{n_\varepsilon} \leq \varepsilon
        \end{align*}
        with $x_{n_\varepsilon} \leq x_n$ and $n \geq n_\varepsilon$ ($\seq{x}{n}$ increasing) and for all $\varepsilon > 0$, take $N(\varepsilon) = n_\varepsilon$.
    \end{proof}
    \item \textbf{Corollary.} A decreasing sequence $\seq{x}{n}$ that is bounded below is convergent.
    \item \textbf{Remark}. There are two cases:
    \begin{enumerate}[label=(\roman*)]
        \item $\seq{x}{n}$ increasing, bounded above $\implies \lim\limits_{n \to \infty} x_n = \sup \set{x_n}{n \in \N}$,
        \item $\seq{x}{n}$ decreasing, bounded below $\implies \lim\limits_{n \to \infty} x_n = \inf \set{x_n}{n \in \N}$.
    \end{enumerate}
    \item \textbf{Example}. Take $x_n = 2^{-n}$ decreasing, bounded below. Then
    $$\lim\limits_{n \to \infty} x_n = \inf\set{2^{-n}}{ n \in \N} = 0.$$
    \item \textbf{Corollary}. Let $\seq{x}{n}$ be a monotone sequence. Then $\seq{x}{n}$ is convergent if and only if $\seq{x}{n}$ is bounded.
\end{itemize}

\subsection{Algebraic Manipulations of Limits}

\begin{itemize}
    \item[]
    \begin{lemma}
        Let $\seq{x}{n}, \seq{y}{n}$ be convergent sequences. Set $\lim_{n \to \infty} x_n = x, \lim_{n \to \infty} y_n = y$. Three things are true:
    \begin{enumerate}[label=(\arabic*)]
        \item The sequence $(x_n + y_n)_{n \in \N}$ is convergent and $lim_{n \to \infty} (x_n + y_n) = x + y$,
        \item The sequence $(x_n \cdot y_n)_{n \in \N}$ is convergent and $\lim_{n \to \infty} (x_n \cdot y_n) = x \cdot y$,
        \item Assume $x_n \neq 0, y_n \neq 0$, then the sequence $(\nicefrac{y_n}{x_n})_{n \in \N}$ is convergent and $\lim_{n \to \infty} (\nicefrac{y_n}{x_n}) = \nicefrac{y}{x}$.
    \end{enumerate}
    \end{lemma}
    \begin{proof}
        of (2). We want to show that $\forall \varepsilon > 0 \ \exists N(\varepsilon) \in \N$ such that
        $$|x_ny_n - xy| \leq \varepsilon \ \forall n \geq N(\varepsilon).$$
        Note that
        \begin{align*}
            |x_ny_n - xy| &= |(x_n - x)y_n + x(y_n - y)| \\
            &\leq |x_n - x||y_n| + |x||y_n - y|
        \end{align*}
        by the triangle inequality.
        \begin{itemize}[label=\smallblacksquare]
            \item \textbf{Case 1: $x \neq 0$}. Because $\seq{y}{n}$ is convergent, there exists $N_1 \in \N$ such that $|y_n - y| \leq \nicefrac{\varepsilon}{2 \cdot |x|} \ \forall n \geq N_1$. Also, $\seq{y}{n}$ is bounded, i.e., $|y_n| \leq M \ \forall n \in \N \ \exists M \geq 1$. \\\\
            Because $\seq{x}{n}$ is convergent, there exists some $N_2 \in \N$ such that
            $$|x - x_n| \leq \frac{\varepsilon}{2M} \ \forall n \geq N_2.$$
            Therefore, for all $\varepsilon > 0$, we set $N_\varepsilon = \max\{N_1, N_2\}$ and we have that
            $$|x_ny_n - xy| \leq \frac{\varepsilon}{2} + \frac{\varepsilon}{2} \leq \varepsilon \ \forall n \geq N_\varepsilon$$
            which ultimately shows that $\lim_{n \to \infty} x_n \cdot y_n = x \cdot y$.
            \item \textbf{Case 2: x = 0}. Obviously it is true that
            $$|x_ny_n - xy| = |x_ny_n| \leq |x_n||y_n|.$$
            Then, consider that
            \begin{itemize}
                \item $\seq{y}{n}$ convergent implies $|y_n| \leq M$, and
                \item $\seq{x}{n}$ convergent to 0 implies $\forall \varepsilon > 0 \ \exists N_1 \in \N$ such that $|x_n| \leq \nicefrac{\varepsilon}{2M} \ \forall n \geq N_1$. Again, we have
                $$x_ny_n - xy| \leq \frac{\varepsilon}{2M} \cdot M \leq \frac{\varepsilon}{2} \leq \varepsilon \ \forall n \geq N_1$$
                which implies that $\lim_{n \to \infty} x_ny_n = 0$.
            \end{itemize}
        \end{itemize}
    \end{proof}
\end{itemize}

\subsection{Comparison Results}

\begin{itemize}
    \item[]
    \begin{lemma}
        Let $\seq{x}{n}$ and $\seq{y}{n}$ be two convergent sequences such that there exists $N_0 \in \N$ such that
    $$x_n \leq y_n \ \forall n \geq N_0.$$
    Then
    $$\lim\limits_{n \to \infty} x_n \leq \lim\limits_{n \to \infty} y_n.$$
    \end{lemma}
    \begin{proof}
        Let $x := \mylim{n}{\infty} x_n, y := \mylim{n}{\infty}$. For the sake of contradiction, assume that $y < x$. Because $\mylim{n}{\infty} x_n = x$ and $\mylim{n}{\infty} y_n = y$, we have that for $\varepsilon = \nicefrac{(x - y)}{4}$, there exists some $N \in \N$ such that
        $$|x_n - x| \leq \varepsilon \quad \text{and} \quad |y_n - y| \leq \varepsilon \ \forall n \geq N.$$
        Note that
        \begin{alignat*}{2}
            |x_n - x| &\leq \varepsilon &&\implies x \leq x_n + \varepsilon \\
            |y_n - y| &\leq\varepsilon && \implies y \geq y_n - \varepsilon.
        \end{alignat*}
        This means that for $n \geq \max\{N_0, N\}$, we have
        \begin{align*}
            y_n \leq y + \varepsilon &= y + \frac{x - y}{4} \\
            &< \underbrace{y + \frac{x - y}{2}}_{(x + y)/2} \\
            &= x - \frac{x - y}{2} \\
            &\leq x - \frac{x - y}{4} \\
            &= x - \varepsilon < x_n
        \end{align*}
        which implies that $y_n < x_n$ which is a contradiction.
    \end{proof}
    \item[]
    \begin{theorem}[Squeeze Theorem]
        Let $\seq{x}{n}$, $\seq{y}{n}$, $\seq{z}{n}$ be sequences such that
    \begin{itemize}[label=\smallblacksquare]
        \item Sequences $\seq{x}{n}$ and $\seq{z}{n}$ are both converging to some limit $L$, and
        \item There exists some $N_0 \in \N$ such that
        $$x_n \leq y_n \leq z_n \ \forall n \geq N_0.$$
        Then $\seq{y}{n}$ is convergent and $\mylim{n}{\infty} = L$.
    \end{itemize}
    \end{theorem}
        \begin{proof}
            Since $\mylim{n}{\infty} x_n = L$ and $\mylim{n}{\infty} z_n = L$, we have that $\forall \varepsilon > 0 \ \exists N(\varepsilon) \in \N$ such that
            $$|x_n - L| \leq \varepsilon \quad \text{and} \quad |z_n - L| \leq \varepsilon \ \forall n \geq N(\varepsilon).$$
            Then, consider $N \geq \max\{N(\varepsilon), N_0\}$ and we have
            \begin{align*}
                L - \varepsilon \leq x_n \leq y_n \leq z_n \leq L + \varepsilon
            \end{align*}
            i.e., $L - \varepsilon \leq y_n \leq L + \varepsilon$ or $|L - y_n| \leq \varepsilon \ \forall n \geq \max\{N(\varepsilon), N_0\} \implies \mylim{n}{\infty} y_n = L$.
        \end{proof}
\end{itemize}

\subsection{Limit Infimum and Limit Supremum}

\begin{itemize}
    \item Let $\seq{x}{n}$ be a bounded sequence. Construct a new sequence $y_n = \{x_k \ : \ k \geq n\}$. We want to show two things about this sequence:
    \begin{enumerate}[label=(\arabic*)]
        \item $\seq{y}{n}$ is decreasing: \\\\
        Consider $n < m$, then we have $y_n = \sup\underbrace{\{x_k \ : \ k \geq n\}}_{A}$ and $y_m = \sup\underbrace{\{x_k \ : \ k \geq m\}}_{B}$. It is clear that $B \subset A \implies \sup(B) \leq \sup(A)$ (by an exercise) which altogether implies that $y_m \leq y_n$.
        \item $\seq{y}{n}$ is bounded below: \\\\
        Since $\seq{x}{n}$ is bounded below, so is $\seq{y}{n}$.
    \end{enumerate}
    \item[]
    \begin{definition}
        Let $\seq{x}{n}$ be a sequence that is bounded above. Then we define the \textbf{limit superior} as
    $$\limsup_{n \to \infty} x_n = \mylimmm{n}{\infty} \left(\underbrace{\sup\{x_k \ : \ k \geq n\}}_{y_n}\right).$$
    Similarly, when $\seq{x}{n}$ is bounded above, we define the \textbf{limit inferior} as
    $$\liminf_{n \to \infty} x_n = \mylimmm{n}{\infty} \left(\underbrace{\inf\{x_k \ : \ k \geq n\}}_{y_n}\right).$$
    \end{definition}
    \item \textbf{Example}. Consider the sequence $x_n = (-1)^n$. Then we have
    \begin{align*}
        \limsup_{n \to \infty} x_n &= \mylimmm{n}{\infty} \sup\{x_k \ : \ k \geq n\} = 1 \\
        \liminf_{n \to \infty} x_n &= \mylimmm{n}{\infty} \inf\{x_k \ : \ k \geq n\} = -1.
    \end{align*}
    Since $\limsup_{n \to \infty} x_n \neq \liminf_{n \to \infty} x_n$, $\seq{x}{n}$ diverges.
    \item[]
    \begin{theorem}
        Let $\seq{x}{n}$ be a bounded sequence. The sequence $\seq{x}{n}$ is convergent if and only if
    $$\limsup_{n \to \infty} x_n = \liminf_{n \to \infty} x_n.$$
    \end{theorem}
    \begin{proof}
        Assume that $\seq{x}{n}$ is convergent and let $x := \mylim{x}{\infty} x_n$. Further, denote
        \begin{align*}
            z_n &= \inf\{x_k \ : \ k \geq n\} \\
            z &= \mylim{n}{\infty} z_n = \liminf_{n \to \infty} x_n.
        \end{align*}
        (we want to show that $x = z$.)
        Since $x_n$ converges to $x$, we know that for every $\varepsilon > 0$, there exists some $N_1(\varepsilon) \in \N$ such that
        $$|x - x_n| \leq \frac{\varepsilon}{2} \ \forall n \geq N_1(\varepsilon)$$
        and since $z_n$ converges to $z$ we similarly have that there exists some $N_2(\varepsilon) \in \N$ such that
        $$|z - z_n| \leq \frac{\varepsilon}{4} \ \forall n \geq N_2(\varepsilon).$$
        Recall that $z_n = \inf\{x_k \ : \ k \geq n\}$. Then, thanks to the characterization of infimums, we know that for every $\varepsilon > 0$ (pick our $\varepsilon = \nicefrac{\varepsilon}{4}$), there exists some $x_{m_\varepsilon} \in \{x_n \ : \ k \geq n\}$ such that
        $$|x_{m_\varepsilon} - z_n| = x_{m_\varepsilon} - z_n \leq \frac{\varepsilon}{4}$$
        since $m_\varepsilon \geq n$ and $z_n \leq x_{m_\varepsilon}$. \\\\
        Now, compute
        \begin{align*}
            |x - z| &= |x - x_{m_\varepsilon} + x_{m_\varepsilon} - z_n + z_n - z| \\
            &\leq \underbrace{|x - x_{m_\varepsilon}|}_{\leq \nicefrac{\varepsilon}{2}} + \underbrace{|x_{m_\varepsilon} - z_n|}_{\leq \nicefrac{\varepsilon}{4}} + \underbrace{|z_n - z|}_{\leq \nicefrac{\varepsilon}{4}} \\
            &\leq \varepsilon \ \forall m \geq n \geq N_1(\varepsilon).
        \end{align*}
        So we proved that $\forall \varepsilon > 0$,
        \begin{align*}
            |x - z| &\leq \varepsilon \\
            \implies x - z &= 0 \\
            \implies x = z.
        \end{align*}
        From a similar argument, $\mylim{n}{\infty} x_n = \limsup_{n \to \infty} x_n = y$, i.e., $x = y \implies y = z$. \\\\
        The other direction is much simpler. Assume $y = \limsup_{n \to \infty} x_n = \liminf_{n \to \infty} x_n = z$. Then, recall
        \begin{alignat*}{2}
            y &\leftarrow y_n &&= \sup\{x_k \ : \ k \geq n\} \\
            z &\leftarrow z_n &&= \inf\{x_k \ : \ k \geq n\}.
        \end{alignat*}
        So, by our assumption we have $z_n \leq x_n \leq y_n$ which by the squeeze theorem implies that $z = y$. Therefore, $x_n \rightarrow y = z$ i.e., $\lim_{n \to \infty} x_n = y = z$.
    \end{proof}
\end{itemize}

\subsection{Subsequences}

\begin{itemize}
    \item[]
    \begin{definition}
        A \textbf{subsequence} $\seq{x}{n}$ is a sequence $\seq{y}{n}$ such that for all $k \in \N$, there exists $n_1, n_2, \ldots, n_k \in \N$ with $n_1 < n_2 < \cdots < n_k$ with $y_k = x_{n_k}$.
    \end{definition}
    \item[]
    \begin{lemma}
        Any subsequence $(x_{n_k})$ of a convergent sequence $\seq{x}{k}$ is convergent to the same limit.
    \end{lemma}
    \begin{proof}
        Left as an exercise.
    \end{proof}
    \item[]
    \begin{lemma}
        Let $(n_k)_{k = 1}^{\infty}$ be a sequence of increasing natural numbers. Then
    $$n_k \geq k.$$
    \end{lemma}
    \begin{proof}
        By induction (left as an exercise).
    \end{proof}
    \begin{proof}
        \textbf{(Of first Lemma)}. For every $\varepsilon > 0$, there exists some $N(\varepsilon) \in \N$ such that
        $$|x - x_k| \leq \varepsilon \ \forall k \geq N(\varepsilon)$$
        (because $x_k \rightarrow x$). From previous Lemma (above), $n_k \geq k \geq N(\varepsilon)$ so we have
        $$|x - \underbrace{x_{n_k}}_{ = y_k}| \leq \varepsilon \ \forall n_k \geq N(\varepsilon).$$
    \end{proof}
    \item[]
    \begin{lemma}
        Any subsequence of a bounded sequence is bounded.
    \end{lemma}
    \begin{proof}
        Left as an exercise.
    \end{proof}
\end{itemize}

\subsection{Bolzano-Weirstrass}

\begin{itemize}
    \item[]
    \begin{definition}
        A \textbf{peak point} of a sequence $\seq{x}{n}$ is a term $x_p$ such that $x_p > q_p$ means $q > p$.
    \end{definition}
    \item[]
    \begin{lemma}[Monotone Subsequence]
        Every sequence has a monotone subsequence.
    \end{lemma}
    \begin{proof}
        We argue on whether the sequence has either (A) no peak point, (B) finitely many peak points, or (C) infinitely many peak points.
        \begin{enumerate}[label=(\Alph*)]
            \item We construct $(x_{n_k})_{k \in \N}$ as follows:
            \begin{alignat*}{2}
                n_1 &&= 1, x_{n_1} &= x_1 \text{ is not a peak point (no peak points)} \\
                && &\implies q>1 \text{ such that } x_q > x_1 > x_{n_1}. \\
                n_2 &&= q, x_{n_2} &= x_q \\
                && &\hspace{6pt}\vdots \\
                && & x_{n_1} \leq x_{n_2} \leq \cdots \text{montonically increasing}
            \end{alignat*}
            \item If $x_p$ is the final peak point ($p$ is the largest index where $x_p$ is a peak point), then take $n_1 = p + 1$, $x_{n_1} = x_{p + 1}$. Then proceed as before.
            \item In this case
            $$p_1 < p_2 < \cdots < p_n < \cdots.$$
            By definition, $x_{p_i} > x_{p_j}$ for $j > i$. So we have a decreasing subsequence taking $x_{n_k} = x_{p_k}$.
        \end{enumerate}
    \end{proof}
    \item[]
    \begin{theorem}[Bolzano-Weirstrass]
        Every bounded sequence has a convergent subsequence.
    \end{theorem}
    \begin{proof}
        We have that $\seq{x}{n}$ implies that $(x_{n_k})_{k \in \N}$ is monotone. Further, $\seq{x}{n}$ bounded implies that $(x_{n_k})_{k \in \N}$ is bounded. So we have a montone and bounded subsequence which is thus convergent.
    \end{proof}
    \item[]
    \begin{lemma}
        Let $\seq{x}{n}$ be a bounded sequence. There is a convergent subsequence $x_{n_k}$ such that
    $$\lim_{k \to \infty} x_{n_k} = \limsup_{n \to \infty} x_n.$$
    \end{lemma}
    \begin{proof}
        We proceed by induction. Let $y = \lim_{n \to \infty} y_n$, $y_n = \sup\{x_k \ : \ k \geq n\}$ such that $y = \limsup_{n \to \infty} y_n$. \\\\
        $P(k)$: There are $n_1 < n_2 < \cdots < n_k$ and $x_{n_1}, x_{n_2}, \ldots, x_{n_k}$ such that
        $$\underbrace{y - \frac{1}{i}}_{\rightarrow y} < x_{n_i} < \underbrace{y + \frac{1}{i}}_{\rightarrow y} \ \forall 1 \leq i \leq k$$
        which implies that $x_{n_i} \rightarrow y$ as well by squeeze theorem. \\\\
        $P(1)$: There is $n_1 \in \N$ and $x_{n_1}$ such that $y - 1 < x_{n_1} < y + 1$. \\\\
        We have $y_n \rightarrow y \implies \exists N \in \N$ such that $|y_n - y| \leq \nicefrac{1}{4} \ \forall n \geq N$. Further we have $y_n = \sup\{x_n \ : \ k \geq n\} \implies \exists x_m, m \geq n$ such that $y_n - x_m \leq \nicefrac{1}{4}$ by the characterization of suprema. Then compute
        \begin{align*}
            |y - x_n| &= |y - y_n + y_n - x_m| \\
            &\leq |y - y_n| + |y_n - x_m| \\
            &\leq \frac{1}{4} + \frac{1}{4} = \frac{1}{2}
        \end{align*}
        so $y - \nicefrac{1}{2} \leq x_m \leq y + \nicefrac{1}{2} \implies y - 1 < x_m < y + 1$. \\\\ 
        Next, assume $P(k)$ is true: There is $n_1 < n_2 < \cdots < n_k$ and $x_{n_1}, x_{n_2}, \ldots, x_{n_k}$ such that
        $$y - \frac{1}{i} < x_{n_i} < y + \frac{1}{i} \ \forall 1 \leq i \leq k.$$
        Since $y_n \rightarrow y$, $\exists N \in \N$ such that $|y - y_n| \leq \nicefrac{1}{(4(k + 1))}$, $n \geq \max\{N, n_{k + 1}\} > N_k$. Further, we have $y_n = \sup\{x_k \ : \ k \geq n\}$ which implies there exists some $m \in \N$ with $m \geq n > n_k$ such that
        $$y_n - x_m < \frac{1}{4(k + 1)}.$$
        Therefore, we have
        \begin{alignat*}{2}
            && |y - y_m| &\leq |y - y_n| + |y_n - x_m| \\
            && &\leq \frac{1}{4(k + 1)} + \frac{1}{4(k + 1)} \\
            && &= \frac{1}{2(k + 1)} < \frac{1}{k + 1} \\
            \implies&& y - \frac{1}{k + 1} &< x_m < y + \frac{1}{k + 1}, n_{k + 1} = m > n_k.
        \end{alignat*}
        Therefore, $P(k + 1)$ is true. Thus, we have shown that $P(k)$ is true for all $k \in \N$ and $x_{n_k} \rightarrow y$ by the squeeze theorem.
    \end{proof}
    \item \textbf{Remark}. Every bounded sequence $\seq{x}{n}$ has a convergent subsequence $(x_{n_k})_{k \in \N}$ such that
    $$\lim_{k \to \infty} x_{n_k} = \liminf_{n \to \infty} x_n.$$
    (Proof is the same).
    \item \textbf{Result} (forgot to mention last class). If $\seq{x}{n}$ is bounded, then every convergent subsequence $\sseq{x}{n}{k}$ satisfies
    $$\liminf_{n \to \infty} x_n \leq \lim_{k \to \infty} x_{n_k} \leq \limsup_{n \to \infty} x_n.$$
    \item \textbf{Example}. Take $x_n = (-1)^n$. We want to find the $\limsup$. \\\\
    Show that $\seq{x}{n}$ is bounded, i.e., $x_n \leq 1 \implies \limsup_{n \to \infty} \leq 1$. Then, consider the subsequence $(x_{2k}) = (1)_{k \in \N}$. Then show that $(x_{2k}) \rightarrow_{k \to \infty} 1 \implies 1 \leq \limsup_{n \to \infty} x_n$. \\\\
    Since we have shown that $\limsup_{n \to \infty} x_n \leq 1 \leq \limsup_{n \to \infty} x_n$, we know that $\limsup_{n \to \infty} x_n = 1$.
\end{itemize}

\subsection{Cauchy Sequences}

\begin{itemize}
    \item[] 
    \begin{definition}
        A sequence $\seq{x}{n}$ is \textbf{Cauchy} if $\forall \varepsilon > 0, \exists N \in \N$ with
    $$|x_m - x_n| \leq \varepsilon \ \forall m, n \geq N.$$
    \end{definition}
    \item \textbf{Example}. Take $x_n = \nicefrac{1}{2}(x_{n - 1} + x_{n - 2}), x_0 = 0, x_1 = 1$. Then we compute
    \begin{align*}
        x_n - x_{n - 1} &= \frac{1}{2}(x_{n - 1} + x_{n - 2}) - x_{n - 1} \\
        &= \frac{1}{2}(x_{n - 2} - x_{n - 1}) \\
        x_2 - x_1 &= -\frac{1}{2} \implies x_2 = -\frac{1}{2} + 1 = \frac{1}{2} \\
        x_3 - x_2 &= \frac{1}{2}(1 - \frac{1}{2}) = \left(\frac{1}{2}\right)^2.
    \end{align*}
    Then, by induction we can show that
    $$x_n - x_{n - 1} = \left(-\frac{1}{2}\right)^{n - 1}.$$
    Now take $m \geq n$ and compute
    \begin{align*}
        x_m - x_n &= x_m - x_{m - 1} + x_{m - 1} - x_{m - 2} \pm \cdots \pm x_{n + 1} - x_n \\
        |x_m - x_n| &\leq \left(\frac{1}{2}\right)^{m - 1} + \left(\frac{1}{2}\right)^{m - 2} + \cdots + \left(\frac{1}{2}\right)^n \\
        &= \left(\frac{1}{2}\right)^n\left(1 + \frac{1}{2} + \left(\frac{1}{2}\right)^2 + \cdots + \left(\frac{1}{2}\right)^{m - n - 1}\right) \\
        &= \frac{1 - \left(\frac{1}{2}\right)^{m - 2}}{1 - \frac{1}{2}} \leq 2 \text{ (by geometric sum)} \\
        &\leq 2^{-n + 1}.
    \end{align*}
    Given $\varepsilon > 0$, if we want $2^{-n + 1} \leq \varepsilon$, compute
    \begin{align*}
        2^{-n + 1} &\leq \varepsilon \\
        -n + 1 &\leq \ln(\varepsilon) \\
        n \geq \ln(\varepsilon) + 1.
    \end{align*}
    Therefore, take $N = \lfloor \ln(\varepsilon) \rfloor + 2$, $n \geq \ln(\varepsilon) + 1$. So, for $m, n \geq N \implies |x_m - x_n| \leq \varepsilon \implies x_n$ Cauchy. In fact, $\lim_{n \to \infty} x_n = \nicefrac{2}{3}$ (details left to reader). \\\\
    \textbf{Warning}: Checking the difference between two subsequent terms of the sequence is not enough to ensure that the sequence is Cauchy. \\\\
    For example, consider $x_n = \sum_{k = 1}^n \nicefrac{1}{k}$ (which is not Cauchy).
    \item[]
    \begin{lemma}
        A sequence $\seq{x}{n}$ is convergent if and only if $\seq{x}{n}$ is Cauchy.
    \end{lemma}
    \begin{proof}
        First, assume that $x_n \rightarrow x$. Therefore, $\forall \varepsilon > 0, \ \exists N \in \N$ such that
        $$|x_n - x| \leq \frac{\varepsilon}{2} \ \forall n \geq N.$$
        Therefore, for $\varepsilon > 0$ and $m, n \geq N$,
        $$|x - x_m| \leq \frac{\varepsilon}{2} \text{ and } |x - x_n| \leq \frac{\varepsilon}{2}.$$
        This implies that
        \begin{align*}
            |x_m - x_n| &= |x - x_m + x - x_n| \\
            &\leq |x - x_n| + |x - x_m| \\
            &\leq \frac{\varepsilon}{2} + \frac{\varepsilon}{2} = \varepsilon \\
            &\implies x_n \text{ Cauchy}.
        \end{align*}
        For the other direction, assume $\seq{x}{n}$ is Cauchy. First, we must prove some claims that will lead us to this proof.
        \begin{enumerate}[label=(\arabic*)]
            \item \textbf{Claim}. If $\seq{x}{n}$ is Cauchy, then $\seq{x}{n}$ bounded.
            \begin{proof}
                This proof is omitted because it is almost identical to the proof as convergent implies bounded.
            \end{proof}
            \item \textbf{Claim}. If $\seq{x}{n}$ has a convergent subsequence $(x_{n_k})_{k \in \N}$, then $\seq{x}{n}$ is convergent (to the same limit).
            \begin{proof}
                To see this, assume there exists some $(x_{n_k})$ convergent subsequence and
                \begin{enumerate}[label=(\roman*)]
                    \item We know $x_n$ is Cauchy, i,e,, $\forall \varepsilon > 0, \ \exists N_1 \in \N$ such that
                    $$|x_m - x_n| \leq \frac{\varepsilon}{2} \ \forall m, n \geq N_1.$$
                    \item There exists an $x \in \N$ such that $\forall \varepsilon > 0, \ \exists N_2 \in \N$ such that
                    $$|x - x_{n_k}| \leq \frac{\varepsilon}{2} \ \forall k \geq N_2.$$
                \end{enumerate}
                Now, we will show that the entire sequence $(x_n)$ converges to $x$. \\\\
                For every $\varepsilon > 0$, set $N = \max{N_1, N_2}$. Then compute
                \begin{align*}
                    |x - x_k| &\leq |x - x_{n_k}| + |x_{n_k} - x_k| \\
                    &\leq \frac{\varepsilon}{2} + \frac{\varepsilon}{2} \ \forall k \geq N.
                \end{align*}
            \end{proof}
        \end{enumerate}
        So, we have
        \begin{alignat*}{2}
            (x_n) \text{ Cauchy } &&\xRightarrow{\text{(1)}} (x_n) &\text{ bounded } \\
            &&\xRightarrow{\text{(B.W.)}} (x_n) &\text{ has a convergent subsequence} \\
            &&\xRightarrow{\text{(2)}} (x_n) &\text{ convergent}.
        \end{alignat*}
    \end{proof}
\end{itemize}

\subsection{Divergent Sequences to $\pm \infty$ (Unbounded Case)}

\begin{itemize}
    \item[]
    \begin{definition}
        We say that a sequence $\seq{x}{n}$ is \textbf{diverging to} $+\infty$ if $\forall M \in \R, \ \exists N \in \N$ such that
    $$x_n \geq M \ \forall n \geq M.$$
    Similarly, we say that a sequence is \textbf{diverging to} $-\infty$ if $\forall M \in \R, \ \exists N \in \N$ such that
    $$x_n \leq M \ \forall n \geq N.$$
    \end{definition}
    \item[]
    \begin{lemma}
        If $\seq{x}{n}$ is converging to $\pm \infty$, then every subsequence is convering to $\pm \infty$ (respectively).
    \end{lemma}
\end{itemize}

\section{Continuous Functions}

\begin{itemize}
    \item[]
    \begin{definition}
        Let $a < b$, $x_0 \in (a, b), f:(a, b) \setminus \{x_0\} \mapsto \R$. We say that $f$ \textbf{has a limit} $l$ at $x_0$ if $\forall \varepsilon > 0 \ \exists \delta > 0$ such that, for $x \in (a, b)$
        $$0 < |x - x_0| \leq \delta \implies |f(x) - l| \leq l.$$
        If $f$ has a limit $l$, we write
        $$\lim_{x \to \infty} f(x) = l.$$
    \end{definition}
    \item[]
    \begin{proposition}
        Let $x_0 \in (a, b), f: (a, b) \setminus \{x_0\} \mapsto \R$. If $f$ has a limit at $x_0$, then the limit must be unique.    
    \end{proposition}
    \begin{proof}
        For the sake of contradiction, assume that $l_1, l_2$ are two limits of $f$ at $x_0$ and further assume that $l_1 < l_2$. Take $\varepsilon = (l_2 - l_1) / 4$. Then we have
        \begin{itemize}[label=\smallblacksquare]
            \item $l_1$ a limit means $\exists \delta_1 > 0$ such that
            $$0 < |x - x_0| \leq \delta_1 \implies |f(x) - l_1| \leq \varepsilon.$$
            \item $l_2$ a limit means $\exists \delta_2 > 0$ such that
            $$0 < |x - x_0| \leq \delta_2 \implies |f(x) - l_2| \leq \varepsilon.$$
        \end{itemize}
        Then, compute for $0 < |x - x_0| \leq \min\{\delta_1, \delta_2\}$
        \begin{align*}
            l_2 - l_1 &\leq |l_2 - f(x)| + |l_1 - f(x)| \\ 
            &\leq \varepsilon + \varepsilon = 2\varepsilon \\
            &= \frac{l_2 - l_1}{2} < l_2 - l_1.
        \end{align*}
        Of course, this is a contradiction so $l_1 = l_2$.
    \end{proof}
    \item \textbf{Example}. Take $f(x) = x^2$. We wish to show that $\lim_{x \to 2} f(x) = 4$. To show, we have $\forall \varepsilon > 0 \ \exists \delta > 0$ such that 
    $$0 < |x - 2| \leq \delta \implies |x^2 - 4| \leq \varepsilon.$$
    Then, compute
    \begin{align*}
        |x^2 - 4| &= \underbrace{|x - 2|}_{\leq \delta}|x + 2| \\
        &\leq \delta(|x| + 2) \\
        &\leq \delta(4 + \delta) \\
        \implies |x| &\leq 2 + \delta
    \end{align*}
    (since $|x| = |x - 2 + 2| \leq |x - 2| + 2$). Now we want $\delta(4 + \delta) \leq \varepsilon$. Further, restrict $\delta \leq 1 \implies \delta^2 \leq \delta$ and $\delta(4 + \delta) \leq 5\delta$. We must now show that $5\delta \leq \varepsilon$. \\\\
    So, $\forall \varepsilon > 0$, set $\delta = \min\{1, \nicefrac{\varepsilon}{5}\}$. Then
    $$0 < |x - 2| \leq \delta \implies |x^2 - 4| \leq 5\delta \leq \varepsilon.$$
    So indeed $\lim_{x \to 2} f(x) = 4$.
    \item \textbf{Example}. Take $f(x) = (x - 1)/(x - 1), x \neq 1$. This function is not defined at $x = 1$, so take $f: (-2, 2) \setminus \{1\} \mapsto \R$. However, we claim still that $\lim_{x \to \infty} f(x) = 1$. We show $\forall \varepsilon > 0$, set $\delta = 1$. Then we have 
    $$0 < |x - 1| \leq \delta \implies |f(x) - 1| = |1 - 1| = 0 \leq \varepsilon.$$
    \item[]
    \begin{proposition}
        Let $x_0 \in (a, b), f: (a, b) \setminus \{x_0\} \mapsto \R$. Then $f$ has the limit $l$ at $x_0$ if and only if for every sequence $\seq{z}{n}$, $z_n \in (a, b) \setminus \{x_0\}$ with $z_n \rightarrow x_0$, then the sequence $(f(z_n))_{n \in \N}$ has
        $$f(z_n) \rightarrow l.$$
    \end{proposition}
    \begin{proof}
        No proof provided in class.
    \end{proof}
    \item \textbf{Example}. Take $f(x) = \nicefrac{x}{|x|}, x \neq 0$. How do we show that this does not have a limit at $x = 0$? We could use the epsilon-delta proof technique, but this could be long and tedious. Instead, consider
    \begin{align*}
        \text{Let } z_n &= -\frac{1}{n} \rightarrow 0 \\
        f(z_n) &= -1 \rightarrow -1 \\
        \text{Let } \widetilde{z_n}& = \frac{1}{n} \rightarrow 0 \\
        f(\widetilde{z_n}) &= 1 \rightarrow 1
    \end{align*}
    $\implies f(x)$ does not have a limit at $x_0$.
\end{itemize}

\newpage

\listoftheorems[]
 
\end{document}
